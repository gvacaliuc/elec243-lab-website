\documentclass[11pt]{article}
\usepackage[margin=.75in]{geometry}
\title{\textbf{ELEC 243 \\
	Signal Processing II: Active Circuits \\
	Lab 4}}
	
\author{\textbf{Student Names}}
\date{\textbf{January 23, 2018}}
\begin{document}
\maketitle

\text{Note(To be deleted): Always write the date as a cardinal number (January 23, 2018 or 23 January 2018), not as an ordinal number(January 23\textsuperscript{rd}, 2018). Put team member names in alphabetical order by last name.}

\newpage

\section{Objective}
Your text here 
\\
\\
\text{Note(To be deleted): Describe the objective of the lab. Use your own words rather than copying phrases from the lab assignment. Make it personal. Build on the objective given in the lab assignment by including your own expectations and objectives. Use personal pronouns such as �we� and �I.�}

\section{Preparation}
Your text here
\\\\
\text{Note(To be deleted): Describe how you prepared for this lab assignment. This may include gathering the materials needed, reading, developing a plan, preparing calculations, writing down questions for the labbies, setting up tables and graphs for the data you will be taking, and even building some of your circuits on your breadboard.}

\section{Materials}
Your text here
\\\\
\text{Note(To be deleted):}
\begin{itemize}
\item \text{List any nonstandard equipment/instruments. You don't need to list the standard bench instruments (i.e., VirtualBench, lab PC, etc.). You can just categorize all of this under one bullet point of "Lab bench \#\#instrument suite" and replace \#\# with the bench number you worked at.   }
\item \text{List software}
\item \text{List components and other materials}
\end{itemize}

\section{Procedure}
Your text here
\text{Note (To be deleted): You may find this section unnecessary, and instead find it makes more sense to put a subsection within each experiment with the procedure. That is fine.\\When you number steps in the procedure, use a number with a period, and use a hanging indent. For example:}
\begin{enumerate}
\item \text{We set the function generator to produce a 1 V p-p, 100 Hz sine wave and we measured the voltage gain. We expected the voltage gain to be ... and the actual result was ...}

\end{enumerate}
\\\\\\
\section{Experiment 4.1: The 741 Op-Amp}
Your text here
\\\\
\text{Note(To be deleted): Describe what you did in this experiment. Make it personal rather than copying phrases and steps from the assignment sheet. Use past tense, since you are describing activities that you have completed. Make it sound like a narrative. Weave in observations, expectations, and results. Include any troubleshooting steps you took. }
\\\\
\text{Here is an example of how you might write about your activities:}
\\\\
\indent\text{In Experiment 4.1, we completed two basic steps: We powered up the op-amp and tested the open-loop response. We expected that the circuit we wired would produce...}
\\\\
\indent\text{The team began the first step by color-coordinating the wires on the breadboard as we wired the bus strips to provide positive power, negative power, and ground buses.}
\\\\
\textbf{\text{Make sure your report answers all the questions in bold in the lab manual}}
\\\\
\text{Keep going in this manner, describing your actions and your thoughts. When you insert a figure, follow these guidelines:}
\begin{itemize}
\item \text{Make the figure a reasonable size. If it is too small, it will be hard to read; if it is too large, it takes up too much space and looks out of place.}
\item \text{Center the figure on the page.}
\item \text{Label the figure with a number and a descriptive caption. Place the label beneath the figure.}
\item \text{Refer to the figure in the text: "Figure 2 illustrates the circuit wiring for the open-loop response."}
\item \text{Put extra space between the figure label and the continuing text. Without extra space, the label and the text run together and readers may be confused.}
\end{itemize}

\section{Experiment 4.2:The Inverting Configuration}
Your text here
\\
\section{Experiment 4.3: Transducer Amplifiers}
Your text here
\\
\section{References}
\\\\
\text{Note (To be deleted): List any external references you may have used while performing the lab and in writing up your lab report.}

\section{Conclusion}
\\\\
\text{Note (To be deleted): Summarize the lab experiments and write a conclusion statement on what the lab taught you, both conceptually and skill-wise. Discuss how the results matched your expectations.}

\section{Errors}
Your text here
\\\\
\text{Note (To be deleted): Discuss potential sources of error, how these errors might be minimized, and ways that the lab could be improved. If you identified faulty equipment during your experiments, please make a note of the serial number of the faulty equipment and/or bench number. If you went through any significant troubleshooting during your lab, summarize the process and any lessons learned from it.}


\end{document}